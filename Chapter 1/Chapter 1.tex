%% LyX 2.0.2 created this file.  For more info, see http://www.lyx.org/.
%% Do not edit unless you really know what you are doing.
\documentclass[12pt,twoside,english]{report}
\usepackage{lmodern}
\renewcommand{\familydefault}{\rmdefault}
\usepackage[T1]{fontenc}
\usepackage[latin9]{inputenc}
\usepackage[a4paper]{geometry}
\geometry{verbose,tmargin=3.5cm,bmargin=3cm,lmargin=3.5cm,rmargin=3cm,footskip=1cm}
\usepackage{fancyhdr}
\pagestyle{fancy}
\setcounter{secnumdepth}{3}
\setcounter{tocdepth}{3}
\setlength{\parskip}{\medskipamount}
\setlength{\parindent}{0pt}
\usepackage{setspace}
\usepackage[authoryear]{natbib}
\usepackage{nomencl}
% the following is useful when we have the old nomencl.sty package
\providecommand{\printnomenclature}{\printglossary}
\providecommand{\makenomenclature}{\makeglossary}
\makenomenclature
\setstretch{1.5}

\makeatletter
%%%%%%%%%%%%%%%%%%%%%%%%%%%%%% User specified LaTeX commands.
\usepackage{fancyhdr}
\pagestyle{fancy}
\fancyhead[RE]{\bfseries \nouppercase\leftmark}
\fancyhead[LO]{\bfseries \nouppercase \rightmark}

\renewcommand{\chaptermark}[1]{%
\markboth{\chaptername 
\ \thechapter\ }{}}

 
\fancyhead[LE,RO]{\bfseries\thepage}
\fancyfoot{}
\raggedbottom
\setlength{\parindent}{8mm}

\makeatother

\usepackage{babel}
\begin{document}

\chapter{Introduction \label{chap:PhD-Introduction}}


\section*{Summary}

This chapter outlines the motivation, context, aim and structure of
the thesis. This thesis demonstrates how the variability of meteorological
variables on various spatial and temporal scales might affect a future,
highly renewable electricity dependent Australia. It is primarily
an investigation of the atmospheric modes that either favorably or
adversely affect an Australia-wide renewable electricity network,
but also illustrates the cost competitiveness of a network that takes
into account such atmospheric modes and the variability they represent.


\section{Motivation\label{sec:PhD-Motivation}}

The need for electricity to be produced via previously underutilised
resources is becoming increasingly important. World energy consumption
is showing no signs of decreasing (global annual electricity consumption
grew by nearly 3.5\% p.a. for the period 1998-2008 \citep{InternationalEnergyAgency2011})
while projections suggest continued growth at an average rate of 2.2\%
p.a. for 2009-2035 \citep{InternationalEnergyAgency2009}. Current
energy production is largely sourced from non-renewable resources
like coal, oil and gas, which are in limited supply. The production
of oil could be peaking or may have even peaked \citep{Sorrell2010}
while the use of fossil fuels is a key contributor to the world\textquoteright{}s
anthropogenic carbon dioxide emissions \citep{Stocker2013}. There
is also mounting evidence that the observed increase in concentration
of anthropogenic carbon dioxide is linked to the warming of the climate
over the 20th century \citep{Stocker2013}. Based on the combination
of dwindling supply of fossil fuels and the likely negative consequences
of further global warming there is currently an enhanced impetus to
investigate electricity production via non-carbon-intensive and renewable
resources. 

With increasing installation of renewable electricity technologies
across Australia the question as to whether or not technologies such
as wind and solar are able to provide stable and reliable electricity
at high levels of penetration is therefore quickly becoming a question
that needs addressing. To date, research supporting the delpoyment
of renewable capacity (often undertaken by the wind/solar farm operators
themselves using software like \citet{Vestas2014} and \citet{Vestas2014a})
has focused on the local conditions that affect single wind farms
or solar stations. Yet the processes in the atmosphere that drive
variability on the large-scale are quite different from those that
drive the small scale, and will determine the optimal mix and geographical
placement of wind and solar farms to meet Australia\textquoteright{}s
electrical energy needs. Recent studies by \citet{Elliston2013} and
\citet{AEMO2013} have certainly addressed the issue of providing
enough renewable generation to meet the electrical needs of Australia.
But the extent to which meteorological phenomena and inherent covariances
in the wind and solar fields might affect the resource placement for
a future RE-dependent Australia is yet to be explored.

There currently appears to be no study in the literature that explores
the covariability of the wind and solar fields over Australia. Yet
the balancing of wind and solar resources across large areas and many
different time scales is likely to be an important consideration in
the future. Specifically, if Australia is to rely on renewable electricity,
knowledge of the influence that common large-scale weather patterns
have in modulating the relationship between wind and solar will be
important. Studies from Sweden (\citealp{Widen2011}), the USA (\citet{Short2013};
\citet{Becker2014}), Denmark (\citet{Becker2014}), Spain (\citealp{Santos-Alamillos2012})
and Europe as a whole (\citet{Andresen2014}) have already shown that
the spatio-temporal balancing of wind and solar resources is an important
consideration for reducing costs and optimising the electricity network.
\citet{Santos-Alamillos2012}, in particular, demonstrated that the
relationship between wind and solar can be significantly impacted
by specific, and often common, weather regimes. The current study
is thus relevant because it addresses the need for a large-scale resources
assessment of renewable electricity in Australia and in doing so assesses
the potential for spatio-temporal balancing of the wind and solar
resources---enabling Australia to contribute to a more sustainable
energy future for the world. 


\section{Context\label{sec:PhD-Context}}

For some time there has been a keen interest in using renewable resources
for the extraction of electricity. Small scale appliances driven by
Photovoltaic (PV\nomenclature{PV}{PhotoVoltaic}) cells have been
in use since the mid-late 20th century \citep{Goetzberger2003} while
it has also become increasingly popular for households to install
PV panels for use by the home (nearly 1 million Australian households
had rooftop PV at the end of 2012 \citep{Council2012}). In Australia,
large scale installations of solar technology are not very common.
In fact, the largest solar installation is a 10MW solar thermal power
plant operating in Western Australia \citep{Council2012a}. Wind power
technology has also undergone a dramatic shift over recent decades;
growing at 22\% p.a. over the last 10 years, wind power is one of
the fastest growing technologies in the world \citep{Council2011}.
More than 30\% of Denmark\textquoteright{}s electricity production
comes from wind power \citep{Council2011}, while in Australia the
state of South Australia has a significant investment in wind power
(almost 25\% penetration \citep{Council2012}). Ultimately though,
the inherently variable nature of renewable electricity production
prevents any single renewable resource from being used for large scale
electrical purposes, without significant and expensive storage capacity. 

To overcome issues associated with the high frequency variability
of electricity sources like wind and solar power, the strategic placement
of resources is often explored as a way of smoothing out net electricity
production. \citet{Archer2007} report significant advantages in geographically
dispersing wind farms so long as in combination the total area covered
by all farms increases. The interconnecting of wind farms does not
guarantee that moments of low output are avoided but instead the combined
output starts to resemble a single farm with steady wind speeds \citep{Archer2007}.
In terms of the solar field an early study by \citet{Wiemken2001}
found that the combined output from 100 grid connected PV systems
in Germany had vastly smoother net output than any of the PV systems
on their own. Indeed more recent studies support such advantages of
geographical dispersion. Decorrelation was found at 6km for grid connect
PV utilising one second data in northern Spain (\citet{Marcos2011}).
In a study encompasing the province of Ontario, Canada, \citet{Rowlands2014}
found decorrelation at 800-100km using hourly simulated PV output.
\citet{Mills2010} found decorellation at 20km for one minute solar
insolation data in the south-central USA, and in Japan \citet{Murata2009}
found decorrelation at either 2km for one minute data or 9km using
five minute data for PV systems that spanned the country. However,
in order to gain further steadiness, or reliability in renewable electricity
output, the combination of different resources like wind and solar
is often explored.

Commonly referred to as a hybrid electricity system, the combination
of wind and solar electricity (along with appropriate back-up energy
supply) is widely accepted as one such hybrid system that is effective
at producing useful electricity for larger scale purposes \citep{Blackburn2010}.
The deficiencies normally associated with either wind or solar are
often overcome when they are combined. Solar radiation at some locations,
for instance, has a diurnal cycle that is qualitatively quite similar
to that of electrical demand (\citealp{Budischak2013}) while wind
power has the distinct advantage of being largely independent from
the diurnal solar cycle (\citealp{AEMO2013a})---with the exception
of desert regions, which can experience night-time maxima in wind
speed due to the nocturnal jet. \citet{Heide2010} found that the
seasonal mix of 55\% wind and 45\% solar was able to meet European
electrical demand while also reducing the need for storage by a factor
of two (when compared to wind-only or solar-only scenarios). Numerous
other studies from countries such as Canada, the United States, China
and Greece have also found clear advantages in combining wind and
solar (\citealp{Hoicka2011}; \citealp{Blackburn2010}; \citealt{Chen2010};
\citealt{Katsigiannis2010}, respectively). 

In the context of Australia, research in recent years is starting
to demonstrate that a large reliance on renewable electricity, perhaps
even 100\%, could be possible in the future without compromising supply
to the network. Using data from the Bureau of Meteorology (BoM) \citet{Wright2010}
have shown that a mixture of 38\% concentrating solar thermal, 44\%
wind power and 18\% other back-up sources, spread-out across the continent,
could supply all of Australia\textquoteright{}s power needs by 2020.
\citet{Elliston2013}, which used BoM and Australian Electricity Market
Operator \nomenclature{AEMO}{Australian Energy Market Operator}(AEMO)
data, suggested the mix for a future 100\% renewable Australia could
be between 46-58\% onshore wind power, 13-22\% Concentrating Solar
Thermal \nomenclature{CST}{Concentrating Solar Thermal }(CST), 15-20\%
PV, 5-6\% Hydro and the remaining 5-6\% gas (fueled by biofuels),
depending on the cost assumptions. \citet{AEMO2013} showed that 100\%
of Australian demand for electricity could be met in the year 2030
by a combination of 20-40\% of installed capacity from PV, 7-35\%
onshore wind, 10-15\% CST, 10-15\% biogas and then smaller amounts
of capacity from other back-up resources (including hydro, geothermal
and biomass). Importantly, both the \citet{AEMO2013} and \citet{Elliston2013}
studies were designed to maintain the existing reliability requirements
of only 0.002\% unserved electricity per year. Other studies suggest
similar results in terms of the high potential for Australia to integrate
large portions of renewable electricity into the national electricity
mix, but with less definitive renewable electricity configurations
(see \citealt{Shafiullah2012}; \citealt{Liu2011}). One might conclude
from a brief survey of the literature that the prospects of dispersing
renewable resources across the Australian continent and achieving
a reliable net electricity output are quite high. However, most studies
in the Australian context use data that are sparse, due to a reliance
on observations, and data that span only a couple of years at most
(due to computational limitations). Much of the variability of a large-scale
renewable network is therefore lost because the data do not resolve
appropriate spatial and temporal scales. In order to properly gauge
the large-scale and long-term renewable electricity potential of Australia
a vastly more comprehensive spatio-temporal investigation is needed. 


\section{Aim \label{sec:PhD-Aim}}

The aim of the current study is to analyse the covariability of the
wind and solar fields across all of Australia in order to reveal the
potential for Australia to extract useful renewable electricity from
both fields. Via a synoptic climatology and a short-term optimisation,
the current study will investigate the long and short term large-scale
variability of both wind speed/power and solar irradiance/power, including
any links to established atmospheric modes of variability for the
region. Specifically, if there are synoptic weather regimes that can
be identified by the climatology as detrimental to wind and/or solar
power output, are these weather regimes also identified as problematic
in the optimisation study? And if not, what are the differences between
the long-term and short-term studies that result in such a discrepancy?

It is therefore the aim of this PhD to undertake a comprehensive investigation
of climatological and meteorological factors that could significantly
influence output in a future, highly renewable electricity dependent,
Australian electricity network. 


\section{Thesis outline\label{sec:Thesis-outline}}

This thesis is made up of two parts. The first part of the thesis
defines a synoptic climatology of the Australian region and explores
the association between common weather regimes and the wind and solar
potential that co-occurs with each regime. Part 2 presents an alternative
approach for examining the influence of synoptic regimes by first
optimising the location and amount of wind and solar electricity as
part of an electricity model (whose aim is to increase the reliance
of the electricity network on wind and solar). Part 2 then uses the
results from Part 1 (the common weather types) to assist in the diagnosis
of the synoptic regimes that lead to poor and/or favourable large-scale
renewable output from the optimised system. The sensitivity of the
results from the optimisation are tested by varying the inputs and
cost assumptions. Following this, there is a discussion of the results
from Parts 1 and 2, how they compare/contrast, as well as reasons
for any similarities/differences. Finally, conclusions from the thesis
are made, including recommendations of future work.
\end{document}
