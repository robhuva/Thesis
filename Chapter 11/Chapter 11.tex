%% LyX 2.0.2 created this file.  For more info, see http://www.lyx.org/.
%% Do not edit unless you really know what you are doing.
\documentclass[12pt,twoside,english]{report}
\usepackage{lmodern}
\renewcommand{\familydefault}{\rmdefault}
\usepackage[T1]{fontenc}
\usepackage[latin9]{inputenc}
\usepackage[a4paper]{geometry}
\geometry{verbose,tmargin=3.5cm,bmargin=3cm,lmargin=3.5cm,rmargin=3cm,footskip=1cm}
\usepackage{fancyhdr}
\pagestyle{fancy}
\setcounter{secnumdepth}{3}
\setcounter{tocdepth}{3}
\setlength{\parskip}{\medskipamount}
\setlength{\parindent}{0pt}
\usepackage{setspace}
\usepackage[authoryear]{natbib}
\setstretch{1.5}

\makeatletter
%%%%%%%%%%%%%%%%%%%%%%%%%%%%%% User specified LaTeX commands.
\usepackage{fancyhdr}
\pagestyle{fancy}
\fancyhead[RE]{\bfseries \nouppercase\leftmark}
\fancyhead[LO]{\bfseries \nouppercase \rightmark}

\renewcommand{\chaptermark}[1]{%
\markboth{\chaptername 
\ \thechapter\ }{}}

 
\fancyhead[LE,RO]{\bfseries\thepage}
\fancyfoot{}
\raggedbottom
\setlength{\parindent}{8mm}

\makeatother

\usepackage{babel}
\begin{document}

\chapter{Future Work and Conclusions\label{chap:Conclusions-Future-Work}}


\section{Future Work}

There are some key research questions not addressed in this thesis
that require further examination. In particular, the following points:
\begin{description}
\item [{\,\,\,\,\,\,1.}] Simulations that include the use of non-weather-dependent
Renewable Electricity (RE) resources (for instance, geothermal, biomass
and pumped hydro).
\end{description}
Point 1 will allow for an increase in the reliance on RE without the
further exposure to atmospheric variability. This is an important
consideration if the aim of a future Australian electricity network
is to rely heavily---potentially 100\%---on RE resources. In general,
a more sophisticated electricity model should be considered for future
work. Such a model would include these weather independent RE resources,
but would also consider conversions of the solar data to power output
and power flow constraints on network design.
\begin{description}
\item [{\,\,\,\,\,\,2.}] An investigation into whether the drivers
of demand and RE output are correlated.
\end{description}
Point 2 is an important aspect of the variability in RE that would
have been addressed in this thesis had time permitted. For instance,
a question that arises from the consideration of point 2 is whether
or not it was possible that some of the high demand days in Part 2
were offset by high output from RE. And then if so, what the meteorological
reasons were for such a relationship.
\begin{description}
\item [{\,\,\,\,\,\,3.}] An investigation into the implications of
projected climate change on resource availability and any changes
to the relationship between wind and solar.
\end{description}
Point 3 arises from the expectation that shifts in the nature and
positioning of weather systems occur in the future due to anthropogenic
climate change. \citet{Purich2013} note an expected southward shift
in the position of the subtropical ridge while \citet{Barnes2013}
note a similar shift in the mid-latitude storm track. With these points
in mind it is reasonable to suggest that some of the weather regimes
analysed in this thesis might also exhibit changes in their mean positioning.
A question that arises from this point would be: does this type of
latitudinal shift in the climatological positioning of weather systems
then alter the resource availability in the NEM region? And if so,
how quickly this happens will be important. A change in resource availability
that occurs too quickly might result in stranded assets. A region
that was once identified as having a superior RE resource might cease
to have such a quality resource. In this case, contributions from
other, or even new installations, would be needed in order to maintain
supply. This last point would be quite costly. 

Another consequence of climate change is the possible increase in
some extremes (\citet{C.B.2012}). Although confidence is low for
a lot of atmospheric extremes one possible change is a likely increase
in tropical cyclone average maximum wind speed (\citet{C.B.2012}).
It would be important to investigate the implication of having more
extreme behaviour in RE output on an electrical network that relies
on RE to meet demand. In particular, similar to point 2, if the extremes
in the weather also cause more demand for electricity (it is considered
very likely that over most land areas the length, frequency and/or
intensity of heat waves to increase by the end of the 21\textsuperscript{st}
century \citet{C.B.2012}). More extreme heat would likely to result
in much higher demand for electricity in order to keep homes and businesses
at comfortable temperatures. Furthermore, would an increase in extreme
RE output put new and significant stresses on the electrical network?
This too could be costly and thus requires investigation. 
\begin{description}
\item [{\,\,\,\,\,\,4.}] Simulations using imperfect demand forecasts
and financial models.
\end{description}
Point 4 would involve the optimiser being fed demand data that contained
a small random component. The random component to demand would imitate
reality and likely result in the use of more back-up resources. Further
to this it would be possible to operate a financial model for each
resource centre---as part of a new simulation. The inclusion of a
financial model would also ensure that each installation was financially
viable and would likely concentrate resources away from the smaller
RE stations.
\begin{description}
\item [{\,\,\,\,\,\,5.}] Analysis of excluded time steps and their
relationship to RE output.
\end{description}
During the process of assigning SOM-based weather types to the 2010-2011
period some hours had no weather type assigned because of an ambiguous
relationship to the SOM. What was not explored in this thesis is the
relationship of these excluded time steps with RE output, and also
RE-Demand. It is possible with further analysis to reveal if some
of these excluded hours were important hours for the NEM (i.e. for
reasons of over or undersupply). Comparison of the results from such
a study with that found in this thesis would be useful in order to
reveal any differences between the influence on RE output from common
weather patterns, as opposed to extremes. Although it should be noted
that over 93\% of time steps had a SOM node match that survived filtering
in the continental case, leaving a small population of extreme samples
to examine.


\section{Conclusions}

By utilising a Self-Organising Map (SOM) of the ERA-Interim 1989-2009
Mean Sea-Level Pressure (MSLP) field over Australia a series of commonly
occurring features of the synoptic scale MSLP were determined. The
time series of MSLP from ERA-Interim was then converted to a time
series of archetypal weather patterns. Some of the common regimes
were then shown to produce concurrently high wind and solar output,
while others were associated with, on average, limited wind and solar
potential. Analysis of just the wind field from ERA-Interim showed
that decorrelation was reached by spatial separation of approximately
1,300-1,400km. 

Following this an electricity model was built. The electricity model
was given wind and solar data for 2010-2011 from the ACCESS-A regional
model and historical demand data from the Australian Electricity Market
Operator. When given the chance to optimise the whole of Australia
the electricity model chose to install most of the RE in remote Western
Australia (WA). Available locations to be optimised were then limited
to the NEM region because no realistic penalty could be applied to
deter the electricity model from using such unrealistic WA locations.
A base NEM-only scenario was then produced. The NEM-only scenario
with standard transmission costs showed some preference for locations
closer to the capital cities. However, the largest wind installation
(northern Queensland) and the only solar installation from the base
NEM-only scenario were still long distances from the nearest demand
centre. 

By conducting optimisations using data from either the Austral summer
and winter it was revealed that some very distinct summer-only and
winter-only features were responsible for the distribution of resources
seen in the base scenario. In particular, the non-use of north Queensland
in the summer-only optimisation showed that the north Queensland wind
resources is a wintertime (dry season) phenomenon. The lack of installed
solar capacity in the wintertime optimisation also showed that the
standard transmission cost was too large a cost impediment to justify
any use of solar photovoltaics in the wintertime. Scenarios of the
whole 2010-2011 period that utilised increased transmission penalties
also revealed that the solar resource, without storage, was not good
enough to justify significantly larger connection costs. The scenarios
with increased connection costs also resulted in the contraction of
wind resources to four distinct regions (north Queensland, Brisbane,
north-west Tasmania and South Australia). The four wind regimes were
then shown to be highly uncorrelated, reinforcing their usefulness
in combining to meet demand, but also providing an optimisation-based
decorrelation length scale of approximately 884km.

While it was possible to conclude from just the analysis in Part 1
that some synoptic regimes had the potential to be problematic for
RE output, integral to the analysis was the optimisation in Part 2.
Part 2 enabled the identification of specific moments in time where
RE output was diminished. Such specifics were impossible to determine
from solely the climatology. In tandem the two parts combine to allow
for both the analysis of conditions that are challenging for RE output
and the synoptic causes. 

In this thesis it has been shown that the potential for Australia
to extract useful electricity from the wind and solar fields is high.
By approaching the question of how the wind and solar resources vary
at various spatio-temporal resolutions from two different approaches
it was concluded that a high reliance on RE in the National Electricity
Market region is possible, but that there were some large-scale features
(synoptic weather patterns and decorrelation length scales) of the
atmosphere over Australia that need to be taken into account when
designing and operating such an RE-based network. 
\end{document}
